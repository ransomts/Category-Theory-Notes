\documentclass{article}
\usepackage[utf8]{inputenc}
\usepackage[english]{babel}

\usepackage{tikz-cd, amssymb, amsthm, subfiles}

\theoremstyle{definition}
\newtheorem{definition}{Definition}

\theoremstyle{remark}
\newtheorem*{note}{Remark}

\newtheorem{proposition}{Proposition}

\newtheorem{theorem}{Theorem}[section]
\newtheorem{corollary}{Corollary}[theorem]
\newtheorem{lemma}[theorem]{Lemma}

\newcommand{\rtarw}{\rightarrow}

\begin{document}

\section{Categories}
\subfile{sections/categories/introduction}

\section{Abstract Structures}
\subfile{sections/abstract_structures/abstract_structures}
\subfile{sections/abstract_structures/initials_terminals}

\section{Duality}
\subfile{sections/duality/duality}
\subfile{sections/duality/equalizers}

% \section{Groups and Categories}

\section{Limits and Colimits}
\subfile{sections/limits_colimits/cones}
\subfile{sections/limits_colimits/prop_5_21}
\subfile{sections/limits_colimits/pullbacks_pushins}

\section{Exponentials}
\subfile{sections/exponentials/exponentials}
\subfile{sections/exponentials/cartesian_closed_categories}

\section{Naturality}
\subfile{sections/naturality/functors}
\subfile{sections/naturality/natural_transformations}
\subfile{sections/naturality/representable_functors}

\section{Misc}
\subfile{sections/misc}

\end{document}
