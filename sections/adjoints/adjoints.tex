\documentclass[../../notes.tex]{subfiles}

\begin{document}

\subsection{Adjoints}

\begin{definition}[Adjoint]

  A adjunction between categoriies C and D consists of functors
  \[
    \begin{tikzcd}
      F : C \arrow[r, shift right] & D : C \arrow[l, shift right]
    \end{tikzcd}
  \]

  with a natural transformation
  \[
    \eta : 1_C \rtarw U \circ F
  \]
  with the property indicated in the diagram below

  \[
    \begin{tikzcd}
      F(C) \arrow[r, dashrightarrow,"g"] & D \\
      U(F(C)) \arrow[r, "U(g)"] & U(D) \\
      C \arrow[u, "\eta_C"] \arrow[ur, "f"]
    \end{tikzcd}
  \]
\end{definition}

%\begin{propositon}
  Every adjoint pair $F \dashv U$ with $ U : D \rtarw C$,
  unit $\eta : UF \rtarw 1_C$ and counit $\epsilon : 1_D \rtarw FU$
  gives rise to a monad $(T, \eta, \mu)$ on C with
  \[ T = U \circ F : C \rtarw C \]
  \[ \eta: 1 \rtarw T\]
  \[ \mu = U_{\epsilon}F : T^2 \rtarw T \]
 % \end{propostion}

\end{document}