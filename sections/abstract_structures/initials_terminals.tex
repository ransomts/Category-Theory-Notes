\documentclass[../../notes.tex]{subfiles}

\begin{document}

\begin{definition}
  An object $0$ is an initial object if for every object A, there is a unique
  map $ 0 \rightarrow A $
\end{definition}

\begin{proposition}
  Initial and terminal objects are unique up to isomorphism.
\end{proposition}
\begin{proof}
  Suppose that $0$ and $0'$ are both terminal or initial objects
  in some category $C$; this diagram states that $0$ and $0'$ are uniquely isomorphic.

  % \[
  %   \begin{tikzcd}
  %     0 \arrow[r, "u"] \arrow[dr, 1_{0}] & 0' \arrow[d, "v"] \arrow["dr", 1_{0'}] &
  %     \\
  %     & 0 \arrow[r, "u"] & 0'
  %   \end{tikzcd}
  % \]
  For terminal objects, apply the previous to $C^{op}$.
\end{proof}

\begin{definition}{Disjoint Union}
  The disjoint union of two sets $A$ and $B$ is the set 
  $$ A \sqcup B = {(0,a):a \in A} \cup {(1,b):b \in B}. $$
\end{definition}

\begin{definition}{Coproduct}
  Let $A$ and $B$ be objects in a category.
  Then a sum (or coproduct) of $A$ and $B$ is an object $A + B$
  together with maps $i_0 : A \rightarrow A + B $ and $i_1 : B \rightarrow A + B $
  such that whenever we have an object $C$ and maps
  $f_0 : A \rightarrow C$ and $f_1 : B \rightarrow C$, there is a unique map
  $f : A + B \rightarrow C$ such that $f_0 = fi_0 $ and $f_1 = fi_1$

  \[
    \begin{tikzcd}
      A \arrow[dr, "f_0"] \arrow[r, "i_0"] & A + B & \arrow[l, "i_1"] \arrow[dl, "f_1"] B \\
      &   C   & 
    \end{tikzcd}
  \]
  
\end{definition}

% TODO: Put in example 3.8 (In the category of proofs...)

\begin{theorem}
  Let $ A $ and $B$ be objects and let $ A \rightarrow["i_0"] P \leftarrow["i_1"] B $ and
  $ A \rightarrow["j_0"] Q \leftarrow["j_1"] B $ be two sums of A and B.
  Then there exists a unique isomorphism $f:P \rightarrow Q $ such that the following diagram commutes:

  \[
    \begin{tikzcd}
      & P \arrow[dd, "f"] &   \\
      A  \arrow[dr, "j_0"] \arrow[ur, "i_0"]  & & B \arrow[ul, "i_1"] \arrow[dl, "j_1"] \\
      & Q                 &   \\
    \end{tikzcd}
  \]
\end{theorem}

% Nathan Jacobson's theorem... what is it?

\end{document}
