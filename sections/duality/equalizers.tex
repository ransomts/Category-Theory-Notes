\documentclass[../../notes.tex]{subfiles}

\begin{document}

\subsection{Equalizers and Coequalizers}
\subsubsection{Equalizers}
\begin{proposition}

  In any category, if $e : E \rtarw A$ is an equalizer of some pair of arrows, then e is monic.

\end{proposition}

\begin{proof}

  Consider the diagram
  \[
    \begin{tikzcd}
      E \arrow[r, "e"] &
      A \arrow[r, shift left, "f"] \arrow[r, shift right, swap, "g"] & B \\
      Z \arrow[u, shift left, "x"] \arrow[u, shift right, swap, "y"]
      \arrow[ur, swap, "z"]& & \\
    \end{tikzcd}
  \]
  in which we assume $e$ is the equalizer of $f$ and $g$. Supposing $ex = ey$, we want to show $ x = y $. Put $z = ex = ey$. Then $fz = fex = gex = gz$, so there is a \textit{unique} $u : Z \rtarw E$ such that $eu = z$. So from $ex = z$ follows that $x = u = y$.
  
\end{proof}

In SETS, the equalizer would just be the set $ { x \in A | f(x) = g(x) } $.

% \begin{example}

Suppose $ f, g : R^2 \rtarw R $ where $ f (x, y) = x^2 + y^2 $
and $ g = 1 $. We take the equalizer, say in TOP, which is the subspace
$ S = { (x, y) \in R^2 | X^2 + y^2 = 1} \rtarw R^2 $ %TODO: change this arrow to monic (hook)
which is the unit circle in the plane!

% \end{example}

Awodey: In abelian groups though, using the fact that $$ f(x) = g(x)$$ iff $$(f-g)(x) = 0 $$ we know that the equalizer of $f$ and $g$ is the same as that of the homomorphism $(f-g)$ and the zero homomorphism $ 0 : A \rtarw B $, so it suffices to consider equalizers of the special form $ A(h, 0) \rightarrowtail A $ for arbitrary homomorphisms $ h: A \rtarw B $. This subgroup of A sis the \textit{kernel}.

Cook: In abelian groups:
\begin{tikzcd}
  G \arrow[r, shift left, "Hom \phi"] \arrow[r, shift right, swap, "f"] & H
\end{tikzcd}
$$ E = \{g \in G | \phi (g) = f(g)\} = \{g \in G | \phi (g) = 1_{+1}\} $$
Is the kernel of a homomorphism by definition, also equalizers don't have to exist.

\subsubsection{Coequalizers}
\begin{tikzcd}
  A \arrow[r, shift left] \arrow[r, shift right] & B \arrow[r, two heads, "c"] \arrow[dr, swap, "z"] & Q \arrow[d, dashrightarrow, "u"] \\
  &&Z
\end{tikzcd}

This is the weakest equivalence relation that forces $f(a)$ relates $g(a) \forall a \in A$
\end{document}
