\documentclass[../../notes.tex]{subfiles}

\begin{document}

\subsection{Epis and Monos}

\begin{definition}

  In any category C, an arrow $$ f : A \rtarw B $$ is
  called a \textit{monomorphism} if given any $ g, h : C \rtarw A$,
  $gh = fh$ implies $g = h$ \\

  \textit{epimorphism} if given any $i,j : B \rtarw D, if = jf$ implies $i = j$

  % TODO: put in diagrams and arrows with fancy heads/tails

\end{definition}

Remember having a left inverse is monic and having a right inverse is epic. Having both makes the mapping an isomorphism. In SETS, the converse of the previous is also true: every mono-epi is iso; but this is not true in the general case.
% TODO: Propostion 2.6

This definition of monomorphism is the category theory equivalent to injective and this definition of epimorphism is the surjective translation.


\begin{proposition}

  A \textit{function} $ f : A \rtarw B$ between sets is monic just in case it is injective. 
  
\end{proposition}

% \begin{definition}
  An object $0$ is an initial object if for every object A, there is a unique
  map $ 0 \rightarrow A $
\end{definition}

\begin{proposition}
  Initial and terminal objects are unique up to isomorphism.
\end{proposition}
\begin{proof}
  Suppose that $0$ and $0'$ are both terminal or initial objects
  in some category $C$; this diagram states that $0$ and $0'$ are uniquely isomorphic.

  % \[
  %   \begin{tikzcd}
  %     0 \arrow[r, "u"] \arrow[dr, 1_{0}] & 0' \arrow[d, "v"] \arrow["dr", 1_{0'}] &
  %     \\
  %     & 0 \arrow[r, "u"] & 0'
  %   \end{tikzcd}
  % \]
  For terminal objects, apply the previous to $C^{op}$.
\end{proof}

\begin{definition}{Disjoint Union}
  The disjoint union of two sets $A$ and $B$ is the set 
  $$ A \sqcup B = {(0,a):a \in A} \cup {(1,b):b \in B}. $$
\end{definition}

\begin{definition}{Coproduct}
  Let $A$ and $B$ be objects in a category.
  Then a sum (or coproduct) of $A$ and $B$ is an object $A + B$
  together with maps $i_0 : A \rightarrow A + B $ and $i_1 : B \rightarrow A + B $
  such that whenever we have an object $C$ and maps
  $f_0 : A \rightarrow C$ and $f_1 : B \rightarrow C$, there is a unique map
  $f : A + B \rightarrow C$ such that $f_0 = fi_0 $ and $f_1 = fi_1$

  \[
    \begin{tikzcd}
      A \arrow[dr, "f_0"] \arrow[r, "i_0"] & A + B & \arrow[l, "i_1"] \arrow[dl, "f_1"] B \\
      &   C   & 
    \end{tikzcd}
  \]
  
\end{definition}

%TODO: Put in example 3.8 (In the category of proofs...)

\begin{theorem}
  Let $ A $ and $B$ be objects and let $ A \rightarrow["i_0"] P \leftarrow["i_1"] B $ and
  $ A \rightarrow["j_0"] Q \leftarrow["j_1"] B $ be two sums of A and B.
  Then there exists a unique isomorphism $f:P \rightarrow Q $ such that the following diagram commutes:

  \[
    \begin{tikzcd}
      & P \arrow[dd, "f"] &   \\
      A  \arrow[dr, "j_0"] \arrow[ur, "i_0"]  & & B \arrow[ul, "i_1"] \arrow[dl, "j_1"] \\
      & Q                 &   \\
    \end{tikzcd}
  \]
\end{theorem}


\begin{definition}{Product}

  In any category $C$, a product diagram for the objects $A$ and $B$
  consists of an object P and arrows
  % \[
  %   \begin{tikzcd}
  %     A & \arrow[l, p_1] P \arrow[r, p_2] & B
  %   \end{tikzcd}
  % \]
  satisfying the universal mapping property: There is some $u: X \rtarw U$
  such that $x_1 = p_1u$ and $x_2 = p_{2}u$. Given any $v: X \rtarw U$, if
  $p_{1}v = x_1$ and $p_{2}v = x_2$ then $v = u$.
\end{definition}

An example:
Let us consider the category of types of the simply typed $\lambda$-calculus.
The $\lambda$ -calculus is a formalism for the specfication and manipulation of functions, based
on the notions of "binding variables" and function evaluation. The relation $ a ~ b $
(usually called $\beta\eta$ -equivalence) on terms is defined to be the equivalence relation
generated by the equations, and the remaining bound variables:
$$ \lambda x.b = \lambda y.b [y/x] (no y in b) $$

The category of types $ C(\lambda ) $ is now defined as follows:
\begin{itemize}
\item Objects: the types
\item Arrows $A \rtarw B$: closed terms $c: A \rtarw B$, identified if $c ~ c'$,
\item Identities $1_A = \lambda x.x (where x : A) $
\item Composition $c \circ b = \lambda x.c(bx)$.
\end{itemize}

\begin{definition}
  A category $C$ is said to have all finite products if it has a terminal object
  and all binary products (and therewith products of any finite cardinality). The
  category $C$ has all (small) products if every set of objects in $C$ has a product.
\end{definition}

\begin{definition}{Slice Category}
  Let $C$ be a category, and $I$ be a $C$-object. Then the category $C / I$, the slice
  category over I, has the following data.
  \begin{itemize}
  \item The objects are pairs (A, f) where $A$ is an object in $C$ and $f: A \rtarw I$ is
    an arrow.
  \item An arrow from $(A, f)$ to $(B, g)$ is an arrow $j: A \rtarw B$ such that $g \circ j = f$ in $C$
  \item The identity arrow on $(A, f)$ is the arrow $1_A : A \rtarw A$.
  \item Given arrows $j: (A, f) \rtarw (B, g)$ and $ k: (B, g) \rtarw (C, h) $, their compostion
    $k \circ j: (A, f) \rtarw (C, h)$ is the arrow $k \circ j : A \rtarw C$.
  \end{itemize}
\end{definition}
\end{document}
