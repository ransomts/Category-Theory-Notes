\documentclass[../../notes.tex]{subfiles}

\begin{document}

\subsection{Equalizers}

\begin{definition}

  Given a pair of categories and parallel morphisms of the shape
  \[
  \begin{tikzcd}
    A \arrow[r, shift left, "f"] \arrow[r, shift right, swap, "g"]& B 
  \end{tikzcd}
\]

The equalizer is the limit. 
  
\end{definition}

This means that for $f : A \rtarw B$ and $g : A \rtarw B$ in a category $C$,
their equalizer is, if it exists
\begin{itemize}
\item an object $eq(f,g) \in C$
\item a morphism $eq(f,g) \rtarw x$
\item such that
  \begin{itemize}
  \item pulled back to $eq(f,g)$ both morphisms become equal
  \item and $eq(f,g)$ is the universal object with this property
  \end{itemize}
\end{itemize}

Examples :
In $C$ = SET, the equalizer of two function of sets is the subset of
elements of $c$ on which both functions coincide
\[ eq(f,g) = {s \in c | f(s) = g(s)} \] 

For $C$ a category with a zero object the equalizer of a morphism
$f : c \rtarw d$ with the corresponding zero morphism is the kernel of $f$.

\begin{proposition}

  A category has equalizers if it has products and pullbacks.
  
\end{proposition}

\begin{proposition}

  If a category has products and equalizers, then it has limits.
  
\end{proposition}



\end{document}