\begin{proposition}

  In ayn category, if $e : E \rtarw A$ is an eualzire of some pair of arrows, then e is monic.

\end{proposition}

\begin{proof}

  Consider the diagram
  \begin{tikzcd}
    E \arrow[r, "e"] & A \arrow[r, shift left, "f"] \arrow[r, shift right, "g"] & B \\
    Z \arrow[u, shift left, "x"] \arrow[u, shift right, "y"] \arrow[ur, "z"]& & \\
  \end{tikzcd}

  in which we assume $e$ is the equalizer of $f$ and $g$. Supposing $ex = ey$, we want to show $ x = y $. Put $z = ex = ey$. Then $fz = fex = gex = gz$, so there is a \textit{unique} $u : Z \rtarw E$ such that $eu = z$. So from $ex = z$ th follows that $x = u = y$.
  
\end{proof}

In SETS, the qualizer would just be the set $ { x \in A | f(x) = g(x) } $.

\\
Awodey: In abelian groups though, using the fact that $$ f(x) = g(x) iff (f-g)(x) = 0 $$ we know that the equalizer of $f$ and $g$ is the same as that of the homomorphism $(f-g)$ and the zero homomorphism $ 0 : A \rtarw B $, so it suffices to consider equalizers of the special form $ A(h, 0) \rightarrowtail A $ for arbitrary homomorphisms $ h: A \rtarw B $. This subgroup of A sis the \textit{kernel}.
\\
Cook: In abelian groups:
\begin{tikzcd}
  G \arrow[r, shift left, "Hom \phi"] \arrow[r, shift right, "f"] & H
\end{tikzcd}
$$ E = {g \in G | \phi (g) = f(g)} = {g \in G | \phi (g) = 1_{+1)} $$
  Is the kernel by definition, also equalizers don't have to exist. 