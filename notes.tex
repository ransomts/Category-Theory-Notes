\documentclass{article}
\usepackage[utf8]{inputenc}
\usepackage[english]{babel}

\usepackage{tikz-cd}
\usepackage{amssymb, amsthm}
\usepackage{blindtext}

\theoremstyle{definition}
\newtheorem{definition}{Definition}

\theoremstyle{remark}
\newtheorem*{note}{Remark}


\begin{document}


\begin{definition}
  A category consists of objects $obj(C)$ and arrows $hom(C)$
\end{definition}

Categories focus on the relation between themselves, the arrows are the important parts. Here is an example category showing function composition with a communative diagram.
\[
  \begin{tikzcd}[row sep=2.5em]
    & A \arrow[dr, "g"] \\
    B \arrow[ur, "f"] \arrow[rr, "g \circ f"] & & C
  \end{tikzcd}
\]

And a quick check of why these are called communative diagrams
$$
h \circ (g \circ f) = (h \circ g) \circ f
$$

\begin{note}
  For any category $C$ there is always an identity arrow $1_C$
  though it would clutter diagrams if it were written every time.
  \begin{tikzcd}
    C \arrow[loop right, "id_C"]
  \end{tikzcd}
\end{note}

There is always one unique identity homomorphism.
$ \exists ! 1_A : A \rightarrow A$

\begin{proof}
  Assume there are two unique identity morphisms from category
  $A$, $1$ and $1'$ as shown in the diagram below.
  \[\begin{tikzcd}
      A \arrow[r, red, shift left, "1"] \arrow[r, blue, shift right, swap, "1'"] & A
    \end{tikzcd}\]
  Then composing these two homomorphisms makes a contradiction.
  $$ 1 = 1 \circ 1' = 1' $$
\end{proof}

\begin{note}
  A homomorphism moving between a category and itself is also know as an endomorphism.
\end{note}

\begin{definition}
  Small and Locally small categories
  \\
  Let $C$ be a category
  \begin{itemize}
  \item if all $hom(C)$'s together form a set, the category is small
  \item if $hom$ are all sets, the category is locall small
  \end{itemize}
\end{definition}

Some examples of categories:
\begin{itemize}
\item SET - The category of all sets is locally small but not small
  $$ \mathcal{P}(\mathbb{X}) = \{A \subseteq \mathbb{X}\} = 2^{\mathbb{X}} $$
  %$$ A \subseteq \mathbb{X} \leftrightsquigarrow \psi_A(x) =
  %\begin{cases}
  %  1   x \in A \\
  %  0   x \notin
  %\end{cases} $$
  $ B^A $ :: All functions from $A \rightarrow B $
\end{itemize}

\end{document}
