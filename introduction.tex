\begin{definition}
  A category consists of objects $obj(C)$ and arrows $hom(C)$
\end{definition}

Categories focus on the relation between themselves, the arrows are the important parts. Here is an example category showing function composition with a communative diagram.
\[
\begin{tikzcd}[row sep=2.5em]
  & A \arrow[dr, "g"] \\
  B \arrow[ur, "f"] \arrow[rr, "g \circ f"] & & C
\end{tikzcd}
\]

And a quick check of why these are called communative diagrams
$$
h \circ (g \circ f) = (h \circ g) \circ f
$$

\begin{note}
  For any category $C$ there is always an identity arrow $1_C$
  though it would clutter diagrams if it were written every time.
  \begin{tikzcd}
    C \arrow[loop right, "id_C"]
  \end{tikzcd}
\end{note}

There is always one unique identity homomorphism.
$ \exists ! 1_A : A \rightarrow A$

\begin{proof}
  Assume there are two unique identity morphisms from category
  $A$, $1$ and $1'$ as shown in the diagram below.
  \[\begin{tikzcd}
  A \arrow[r, red, shift left, "1"] \arrow[r, blue, shift right, swap, "1'"] & A
  \end{tikzcd}\]
  Then composing these two homomorphisms makes a contradiction.
  $$ 1 = 1 \circ 1' = 1' $$
\end{proof}

\begin{note}
  A homomorphism moving between a category and itself is also know as an endomorphism.
\end{note}

\begin{definition}
  Small and Locally small categories
  \\
  Let $C$ be a category
  \begin{itemize}
  \item if all $hom(C)$'s together form a set, the category is small
  \item if $hom$ are all sets, the category is locall small
  \end{itemize}
\end{definition}

Some examples of categories:
\begin{itemize}
\item SET - The category of all sets with mappings between them is locally small but not small
  $$ \mathcal{P}(\mathbb{X}) = \{A \subseteq \mathbb{X}\} = 2^{\mathbb{X}} $$
  %$$ A \subseteq \mathbb{X} \leftrightsquigarrow \psi_A(x) =
  %\begin{cases}
  %  1   x \in A \\
  %  0   x \notin
  %\end{cases} $$
  $ B^A $ :: All functions from $A \rightarrow B $
\item Grp - An object is a group and a map $ G \rightarrow H $ is a group homomorphism
\item Vect - An object is a vector space and a map $ V \rightarrow W $ is a linear map
\end{itemize}

\begin{definition}
  Let $A$ and $B$ be objcets in a category. Them a map $f : A \rightarrow B$ is an
  isomorphism is the is a map $ f^{-1}  : B \rightarrow A $ (the inverse of f) such that
  $f^{-1} \circ f = Id_A $ and $f \circ f^{-1} = Id_B$.

  If there exists an isomorphism between $A$ and $B$,
  we say that $A$ and $B$ are isomorphic and write $A \cong B$.
\end{definition}

\begin{proposition}
  In Set, a map is an isomorphism iff it is a bijection. Two sets are isomorphic
  off they have the same cardinality.
\end{proposition}

