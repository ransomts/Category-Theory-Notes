\documentclass[../../notes.tex]{subfiles}

\begin{document}

\subsection{Duality}
\begin{proposition}{Formal Duality}
  For any sentence $\Sigma$ in the language of category theory, if $\Sigma$
  follows from the axioms for categories, then so does its dual $\Sigma^*$:

  $$ CT \Rightarrow \Sigma implies CT \Rightarrow \Sigma^* $$
\end{proposition}

Taking a diagram to illustrate, if this is a statement $\Sigma$
\[
  \begin{tikzcd}
    A \arrow[r, "f"] \arrow[dr, "g \circ f"] & B \arrow[d, "g"] \\
    & C
  \end{tikzcd}
\]

then this is the dual statement $\Sigma^*$
\[
  \begin{tikzcd}
    A  & B \arrow[l, "f"] \\
    & C \arrow[ul, "f \circ g"] \arrow[u, "g"]
  \end{tikzcd}
\]
Note how close this is to the idea of an opposite category $C^{op}$.

\begin{proposition}{Conceptual duality}
  For any statement $\Sigma$ about categories, if $\Sigma$ holds for
  all categories, then so does the dual statement $\Sigma^*$.
\end{proposition}
\end{document}
