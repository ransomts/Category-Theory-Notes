\documentclass[../../notes.tex]{subfiles}

\begin{document}

\section{Exponential}

\begin{definition}
   For a cotegory C with binary products, an exponential $C^B$
   is associated with two objects and an evaluation arrow
   $ \epsilon : A \times B \rtarw $ if
  
  \begin{tikzcd}
    A \times B \arrow[r, "f"] & C
  \end{tikzcd}
  
  then there exists a unique

  \begin{tikzcd}
    A \times B \arrow[r, "\tilde{f}"] & C^B
  \end{tikzcd}
  
  where $\tilde{f}$ is the transpose of $f$ such that
  \[
  \begin{tikzcd}
    C^B \times B \arrow[r, "\varepsilon"] & C \\
    A \times B \arrow[u, "\tilde{f} \times 1_{B}"] \arrow[ur, "f"]
  \end{tikzcd}
  and
  \begin{tikzcd}
    C^B \\
    A \arrow[u, "g"]
  \end{tikzcd}
\]
\end{definition}

Check the transpose of the transpose is the thing...

    $$ \varepsilon \circ f \times 1_B = \tilde{f} $$ 
    $$ \bar{\tilde{f}} = \varepsilon \circ ( \tilde{f} \times 1_B ) = f $$ 
    $$ \bar{g} = \varepsilon \circ (g \times 1_B ) \therefore \tilde{\bar{g}} = g $$


In set:
$$ C^B = \{ f : B \rtarw C \} $$
$$ \varepsilon : C^B \times B \rtarw C $$
$$ \varepsilon (f, b) = f(b) $$
So...
$$ f : A \times B \rtarw C $$
$$ f(a, b) = c $$
$$ \tilde{f(a)} = f(a, *) : B \rtarw C $$
$$ g : A \rtarw C^B $$
$$ \bar{g} : A \times B \rtarw C $$
$$ \bar{g}(a, b) = (g(a))(b) $$
This is just currying!

\subsection{More Exponentials}

Another way to write $ A \times B \rtarw C $ is as $ A \rtarw C^B $,
meaning $Hom(A \times B, C) \cong Hom(A, C^B) $ is natural. This is an example of
a left adjoint ($- \times B$) and a right adjoint ($ -^B $). 

TODO: Example 6.6 with graphs pg 124

%\begin{proposition}

  $ -^A : C \rtarw C$ is a functor.
For a functor we need to know what it does with objects and with morphisms.
%  \begin{proof}

    Define $ \beta : B^A \rtarw C^A $ and $ \varepsilon : <TODO> $.
    then
  \begin{tikzcd}
    B \arrow[|->, r] & B^A &
    B \arrow[r, "\beta"] & C &
    B^A \arrow[r, "\beta \circ \varepsilon"] & C^A 
  \end{tikzcd}

%  \begin{tikzcd}
    
%    A \arrow[r, "\varepsilon"] & C \\
%    B^A \times A \arrow[u, "\bar{\beta \circ \varepsilon} \times 1_A"]
%    \arrow[ur, "\beta \circ \varepsilon"] &
    
%  \end{tikzcd}
    
%  \end{proof}
  
%\end{propostion}
\end{document}
